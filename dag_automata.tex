% This must be in the first 5 lines to tell arXiv to use pdfLaTeX, which is
% strongly recommended.
\pdfoutput=1
% In particular, the hyperref package requires pdfLaTeX in order to break URLs
% across lines.

\documentclass[11pt]{article}

% Remove the "review" option to generate the final version.
\usepackage[]{ACL2023} 
% Standard package includes
\usepackage{times}
\usepackage{latexsym}
\usepackage{hyperref}
\usepackage{tikz-cd}
% For proper rendering and hyphenation of words containing Latin characters
% (including in bib files)
\usepackage[T1]{fontenc}
% This assumes your files are encoded as UTF8
\usepackage[utf8]{inputenc}

% This is not strictly necessary, and may be commented out. However, it will
% improve the layout of the manuscript, and will typically save some space.
\usepackage{microtype}

% This is also not strictly necessary, and may be commented out. However, it
% will improve the aesthetics of text in the typewriter font.
\usepackage{inconsolata}


%%%%%%%%%%%%%%%%%%%%%%%%%%%%%%%%%%%%%%%%%%%%%%%%%%%%%%%%%%%%%%%%%%%%%%%%%%%%%%%%%%%%%%%%%%%%%%%%%%%
% egalpaya_defs.tex                                                                               %
% A collection of common packages and definitions shared throughout documents                     %
%%%%%%%%%%%%%%%%%%%%%%%%%%%%%%%%%%%%%%%%%%%%%%%%%%%%%%%%%%%%%%%%%%%%%%%%%%%%%%%%%%%%%%%%%%%%%%%%%%%

\usepackage{blindtext}
\usepackage{enumitem}
\usepackage{caption}
\usepackage{placeins}
\usepackage{graphicx}
\usepackage{fancybox}
\usepackage{hyperref}

\usepackage{amsmath,amsfonts,amsthm,amssymb,xcolor}
% \let\amsboldsymbol\boldsymbol % if you want to see the bad
% boldsymbol for comparison
\usepackage{bm} % fixes boldsymbol
\usepackage{bbm}

\usepackage{nicefrac}

\usepackage{float}

\usepackage[ruled,vlined, linesnumbered]{algorithm2e}
\SetKwComment{Comment}{/* }{ */}


\def\calA{\mathcal{A}}
\def\calC{\mathcal{C}}
\def\calD{\mathcal{D}}
\def\calE{\mathcal{E}}
\def\calF{\mathcal{F}}
\def\calG{\mathcal{G}}
\def\calH{\mathcal{H}}
\def\calL{\mathcal{L}}
\def\calQ{\mathcal{Q}}
\def\calS{\mathcal{S}}
\def\calT{\mathcal{T}}
\def\calM{\mathcal{M}}

\def\aa{\pmb{\mathit{a}}}
\newcommand\bb{\boldsymbol{\mathit{b}}}
\newcommand\cc{\boldsymbol{\mathit{c}}}
\newcommand\dd{\boldsymbol{\mathit{d}}}
\newcommand\ee{\boldsymbol{\mathit{e}}}
\newcommand\ff{\boldsymbol{\mathit{f}}}
\renewcommand\gg{\boldsymbol{\mathit{g}}}
\newcommand\ii{\boldsymbol{\mathit{i}}}
\newcommand\jj{\boldsymbol{\mathit{j}}}
\newcommand\kk{\boldsymbol{\mathit{k}}}
\renewcommand\ll{\boldsymbol{\mathit{l}}}
\newcommand\pp{\boldsymbol{\mathit{p}}}
\newcommand\qq{\boldsymbol{\mathit{q}}}
\newcommand\bs{\boldsymbol{\mathit{s}}}
\newcommand\nn{\boldsymbol{\mathit{n}}}
\newcommand\rr{\boldsymbol{\mathit{r}}}
\renewcommand\ss{\boldsymbol{\mathit{s}}}
\def\tt{\boldsymbol{\mathit{t}}}
\newcommand\uu{\boldsymbol{\mathit{u}}}
\newcommand\vv{\boldsymbol{\mathit{v}}}
\newcommand\ww{\boldsymbol{\mathit{w}}}
\newcommand\yy{\boldsymbol{\mathit{y}}}
\newcommand\zz{\boldsymbol{\mathit{z}}}
\newcommand\xx{\boldsymbol{\mathit{x}}}

\renewcommand\AA{\boldsymbol{\mathit{A}}}
\newcommand\BB{\boldsymbol{\mathit{B}}}
\newcommand\CC{\boldsymbol{\mathit{C}}}
\newcommand\DD{\boldsymbol{\mathit{D}}}
\newcommand\EE{\boldsymbol{\mathit{E}}}
\newcommand\GG{\boldsymbol{\mathit{G}}}
\newcommand\HH{\boldsymbol{{H}}}
\newcommand\II{\boldsymbol{\mathit{I}}}
\newcommand\JJ{\boldsymbol{\mathit{J}}}
\newcommand\KK{\boldsymbol{\mathit{K}}}
\newcommand\NN{\boldsymbol{\mathit{N}}}
\newcommand\MM{\boldsymbol{\mathit{M}}}
\newcommand\LL{\boldsymbol{\mathit{L}}}
\newcommand\PP{\boldsymbol{\mathit{P}}}
\newcommand\QQ{\boldsymbol{\mathit{Q}}}
\newcommand\RR{\boldsymbol{\mathit{R}}}
\renewcommand\SS{\boldsymbol{\mathit{S}}}
\newcommand\TT{\boldsymbol{\mathit{T}}}
\newcommand\UU{\boldsymbol{\mathit{U}}}
\newcommand\WW{\boldsymbol{\mathit{W}}}
\newcommand\VV{\boldsymbol{\mathit{V}}}
\newcommand\XX{\boldsymbol{\mathit{X}}}
\newcommand\YY{\boldsymbol{\mathit{Y}}}

\newcommand\veczero{\boldsymbol{0}}
\newcommand\vecone{\boldsymbol{1}}

\newcommand\R{\mathbb{R}}
\newcommand\N{\mathbb{N}}
\newcommand\K{\mathbb{K}}

\newtheorem{theorem}{Theorem}[section]
\newtheorem{corollary}[theorem]{Corollary}
\newtheorem{lemma}[theorem]{Lemma}
\newtheorem{observation}[theorem]{Observation}
\newtheorem{proposition}[theorem]{Proposition}
\newtheorem{claim}[theorem]{Claim}
\newtheorem{fact}[theorem]{Fact}
\newtheorem{assumption}[theorem]{Assumption}
\newtheorem{warning}[theorem]{Warning}
\newtheorem{conjecture}[theorem]{Conjecture}

\newtheorem*{theorem*}{Theorem}
\newtheorem*{corollary*}{Corollary}
\newtheorem*{conjecture*}{Conjecture}
\newtheorem*{lemma*}{Lemma}
\newtheorem*{thm*}{Theorem}
\newtheorem*{prop*}{Proposition}
\newtheorem*{obs*}{Observation}

\newtheorem*{example}{Example}
\newtheorem*{remark*}{Remark}
\newtheorem*{rec*}{Recommendation}


\newtheorem{definition}[theorem]{Definition}
\newtheorem{remark}[theorem]{Remark}
\newtheorem*{definition*}{Definition}

\newcommand\DDelta{\boldsymbol{\mathit{\Delta}}}
\newcommand\Ppsi{\boldsymbol{\mathit{\Psi}}}
\newcommand\PPsi{\boldsymbol{\mathit{\Psi}}}
\newcommand\ppsi{\boldsymbol{\mathit{\psi}}}
\newcommand\pphi{\boldsymbol{\mathit{\phi}}}
\newcommand\PPhi{\boldsymbol{\Phi}}
%\newcommand\Llambda{\boldsymbol{\mathit{\Lambda}}}
\newcommand\LLambda{\boldsymbol{\mathit{\Lambda}}}
\newcommand\PPi{\boldsymbol{\Pi}}

\newcommand\ppi{\boldsymbol{\pi}}
\newcommand\cchi{\boldsymbol{\chi}}
\newcommand\aalpha{\boldsymbol{\alpha}}
\newcommand\bbeta{\boldsymbol{\beta}}
\newcommand\ggamma{\boldsymbol{\gamma}}
\newcommand\ddelta{\boldsymbol{\delta}}

\newcommand\rrho{\boldsymbol{\rho}}
\newcommand\xxi{\boldsymbol{\xi}}
%\newcommand\cchi{\boldsymbol{\chi}}

% If the title and author information does not fit in the area allocated,
% uncomment the following
%
%\setlength\titlebox{<dim>}
%
% and set <dim> to something 5cm or larger.
\setcitestyle{numbers,square}

\title{Analysis and Properties of Weighted DAG Automata}


\author{ Eishitha Galpayage Don \\
  ETH Zurich \\
  \texttt{egalpayage@student.ethz.ch} \\}

\begin{document}
\maketitle
\begin{abstract}
In this paper, we study the weighted DAG automata of \citet{chiang2018weighted},
a recent formalism proposed to model semantic graphs. \citet{blum2019properties}
establish many properties and algorithms of these automata for the unweighted
case, such as closure properties, determinisation and minimisation. However,
generalisations to the weighted case remains an open area of research, and this
paper aims to start addressing these gaps. Indeed, algorithms and properties for
weighted automata are important for future practical applications, such as the
construction of probability models and inferences. In this paper, we will
formalise and prove the recognition algorithm of \cite{chiang2018weighted}, give
algorithms for computing the weighted intersection and union, as well as provide
a set of definitions and analysis methods in line with existing literature on
weighted automata. 
\end{abstract}

\section{Introduction}
String languages, grammars and automata have been extensively studied for
decades, with standardised definitions, properties and algorithms. A core
motivator of such research is the widespread applicability to language syntax
and parsing, for both programming and natural languages. However, another
important goal is the modeling of semantics. For instance, how do we represent
the meaning of an English sentence? Abstract meaning representation
\cite{banarescu2013abstract} is a potential way of modelling meaning via
semantic graphs. Formally, these semantic graphs are labelled, directed, acyclic
graphs (DAGs). In order to describe and parse such structures, we require
formalisms for DAG languages, of which many have been proposed
\cite{kamimura1981parallel, quernheim2012dagger, priese2007finite}, each with
various idiosyncrasies.

The main goal of this project is to investigate theoretical properties and
algorithms admitted by weighted Directed Acyclic Graph (DAG) languages and
automata. The existing literature for weighted automata over strings and trees
share many common ideas and concepts. For instance, forward and backward
weights have been defined for both string and tree automata, representing
essentially the sum of paths leading to or emanating from a given state. This
allows us to compute marginals for transitions, and convert the weight
distribution of transitions emanating from a state into a probability
distribution. Such concepts have applications in machine learning and
probabilistic inference. \cite{chiang2018weighted} outlines a related notion of
`partial analysis' for nodes in a DAG, and we aim to formalise this in a similar
fashion to forward and backward weights. 

There exists ample evidence in the literature to demonstrate that this topic is
still present in contemporary research and there is some potential for extending
the existing frameworks. For example, \citeauthor{blum2019properties} explores
language-theoretic properties of DAG languages and provide an explicit analysis
of \citeauthor{chiang2018weighted}'s formalism. In general however, the papers
focused on the unweighted
case. One of our aims is to begin the process of generalising these results and
definitions to the weighted
case.

\section{Literature review}

\subsection{Semantic Graphs}
The Penn Treebank is a corpus of English natural language sentences paired with
syntactic trees, and is widely used in syntactic modelling. However, the
semantics of a sentence often involves more complex relations that cannot be
modelled as simple rooted trees. Abstract Meaning Representation
\cite{banarescu2013abstract} is an attempt to formalise a system for semantic
representation, in which thousands of English sentences are broken down into
their semantics and expressed as a semantic graph. Semantic graphs as used in
AMR are specified as rooted, directed, acyclic graphs
with both node (leaf) and edge labels
\footnote{\url{https://www.isi.edu/~ulf/amr/help/amr-guidelines.pdf}}. It also
aims to be independent of syntax, e.g ``he described her as a genius'' has an
identical semantic graph to ``she was a genius according to his description''. 

\subsection{DAG Automata and Grammars}
The concepts of DAG automata and grammars have existed for a while, the most
cited being those of \citet{kamimura1981parallel}, \citet{quernheim2012dagger},
and \citet{priese2007finite}. These have all been born out of different needs
and use cases, and as such have quite different properties. For instance, the
formalism of \cite{kamimura1981parallel} recognises only planar DAGs, and is
hence unsuitable for modelling semantic graphs. 

More recently, there have been two main formalisms proposed specifically with
semantic modelling in mind. These are the weighted DAG automata of
\cite{chiang2018weighted}, which are studied in this paper, and the order
preserving DAG grammars of \cite{ericson2019order}. The formalisms differ in
that the DAG automata recognise only vertex labelled
graphs, while OPDGs generate edge labelled hypergraphs, but both
formalisms provide simple transformations converting to and from semantic
graphs. 

\subsubsection{Weighted DAG Automata}
The seminal paper by \cite{chiang2018weighted} describes the DAG automaton
as a simplified version of the formalism proposed by
\cite{quernheim2012dagger}. Most notably, the authors provide proofs of a
number of properties. For instance, they posit that path languages of DAG
automata are regular under the constraint that there exist multiple roots.
Furthermore, they demonstrate that hyperedge replacement graph languages are
closed under intersection with languages recognised by DAG automata. They also
show that testing for emptiness of DAG automata is decidable under the
definition assumed in the present paper, but not under the original definition
by \cite{kamimura1981parallel}, and that general recognition is
NP-complete, even in the non-uniform case (via a reduction from SAT). It is also
interesting to mention that the authors define `extended weighted DAG automata',
which allow rules that can process nodes with unbounded degree. Essentially,
these rules/transitions use a form of regular expressions to match structures.

\cite{blum2019properties} then analyses the language theoretic properties of
these unweighted DAG automata. Namely, they show that recognisable/regular DAG languages
(DAG languages recognised by DAG automata) are closed under union and
intersection, but not under complement. It is also shown that deterministic DAG
automata are strictly less powerful than their nondeterministic counterparts. A
pumping lemma is then presented for regular DAG languages, as well as a
characterisation of infinite DAG languages in terms of cyclic automata. 

Finally, for deterministic DAG automata, \cite{blum2019properties} presents a
definition of equivalent and distinguishable states, a minimisation procedure,
and a polynomial time method for computing the equivalence of deterministic DAG
automata. 

\subsubsection{Order Preserving DAG Grammars}
Weighted order-preserving DAG grammars
(WOPDGs) on the other hand are a generative model.

The formalism for OPDGs introduced by \cite{ericson2019order} is based on
introducing several constraints on the original definition of weighted DAG
grammars. OPDGs have been subsequently studied in a series of papers. For
instance, \cite{ericson1opdl} gives a polynomial time algorithm for parsing
OPDGs and mention NP-hard result for grammars obeying a subset of the original
restrictions. \cite{ericson2opdlregularity} establishes an algebraic
characterisation of OPDLs, defining operations such as concatenation, where the
languages induced by the algebra are exactly the languages generated by OPDGs.
This allows for Nerode congruence and the Myhill-Nerode theorem.

Further properties are explored in \cite{ericson3opdlweighted}, who define
weighted OPDLs over commutative semirings. Further, a link between WOPDGs and
weighted tree grammars is established. This allows the minimisation problem for
deterministic WOPDGs to be reduced to the case for a weighted tree grammar. They
then give asymptotic bounds for the complexity of minimisation. OPDLs are also
proved to be MSO-definable (that is, given a grammar, we can construct a monadic
second-order formula such that if $g$ is generated by the grammar, the formula
is a logical consequence of $g$). However, they do not show the reverse: given
an MSO formula that defines a graph language (that is within our `universe'), we
can construct an OPDG that represents the same language. 

As for parsing, \cite{ericson1opdl} proposes a uniform
polynomial parsing algorithm for a generalisation of OPDGs. Uniform is used here
in the sense that the grammar is considered as a part of the input, while
non-uniformity assumes a fixed grammar. Finally, \cite{parsing2} extends
the uniform polynomial parsing algorithm to the weighted case.


\section{Automata Definitions and Preliminaries}
We now present the definition of DAG automata as per \cite{chiang2018weighted}
and \cite{blum2019properties}, starting with some preliminaries.

An \textit{alphabet} $\Sigma$ is a finite, nonempty set of symbols. A
\textit{ranked alphabet} is a tuple $(\Sigma, rank)$, where $\Sigma$ is an
alphabet, and $ rank: \Sigma \rightarrow \mathbb{N} $ is a ranking function. A
\textit{doubly ranked} alphabet has a ranking function $ rank: \Sigma
\rightarrow \N \times \N$, where each symbol is assigned a \textit{head rank}
and a \textit{tail rank}. 

Given a set $S$, a \textit{multiset} over $S$ is a mapping $\mu: S \rightarrow
\mathbb{N}$. Informally, it is a set where elements can appear multiple times,
and we will simply write it as $\{x_1, x_2, \ldots, x_k\}$ where the $x_i$ need
not be distinct. We denote the collection of all multisets over set $S$ as
$\mathcal{M}(S)$. 

A \textit{commutative semiring} is a tuple $\langle \K, \oplus, \otimes,
\veczero, \vecone \rangle$ such that $(\K, \oplus)$ is a commutative monoid with
identity $\veczero$, $(\K, \otimes)$ is a commutative monoid with identity
$\vecone$, $\otimes$ left and right distributes over $\oplus$, and $\otimes$
with $\veczero$ annihilates $\K$. In this paper, when we refer to semirings, we
mean commutative semirings and we will simply denote the structure by $\K$.
Further, a semiring is \textit{zero sum free} if there are no elements $a,b \in
\K \setminus \{\veczero\}$ such that $a \oplus b = \veczero$, and a semiring is
\textit{zero divisor free} if there are no elements $a,b \in \K \setminus
\{\veczero\}$ such that $a \otimes b = \veczero$.

\subsection{$\Sigma$-Graphs and DAGs}
\begin{definition}[$\Sigma$-Graph]\label{def:automatagraph} Let $\Sigma$ be a
  doubly ranked alphabet of node labels. A $\Sigma$-graph over is a tuple $G =
  (V, E, lab_V, in, out)$ such that:
  \begin{itemize}
    \item $V$ and $E$ are finite sets of vertices and edges respectively.
    \item $lab_V: V \rightarrow \Sigma$ is a node labelling function.
    \item $in: V \rightarrow E^*$ is a function mapping each vertex to an
    ordered sequence of incoming edges (possible empty).
    \item $out: V \rightarrow E^*$ is a function mapping each vertex to an
    ordered sequence of outgoing edges (possible empty).
  \end{itemize}

  We additionally require that $(|in(v)|,|out(v)|) = rank(lab_V(v))$, i.e the
  node's incoming/outgoing edges match the rank of the label. Moreover, we
  require that every edge $e \in E$ has $e \in in(v)$ for exactly one $v$, and
  $e \in out(v')$ for exactly one $v'$, and occurs in these two sequences
  exactly once. We refer to $v,v'$ as $tar(e),src(e)$ respectively. 
\end{definition}
From this point on we shall refer to $\Sigma$-graphs as just graphs. 

We say that a node $v$ is a \textit{root} node if $in(v) = \lambda$, where
$\lambda$ represents the empty sequence. Likewise, a \textit{leaf} node $v$ has
$out(v) = \lambda$.

\begin{definition}[Graph Fragment]\label{def:graphfragment} We define a
  $r$-graph fragment (with $r=(h,t)$) as a tuple $\dd_r =
  (V,E,lab_V,in,out,E_h,E_t)$ where $(V,E,lab_V,in,out)$ is a $\Sigma$-graph.
  $E_h = e_1e_2\ldots e_h$ is an ordered set of `head' edges with no source, but
  a target in $V$, and this set is disjoint from $E$. $E_t = e'_1e'_2\ldots
  e'_t$ is likewise an ordered set of `tail' edges with no target, again
  disjoint from both $E$ and $E_t$. A $(0,0)$-graph fragment is just a
  $\Sigma$-graph.
\end{definition}

Intuitively, an $r$-graph fragment is a graph with $h$ `dangling' incoming edges
and $t$ `dangling' outgoing edges for $r = (h,t)$. 

A \textit{path} in a graph or graph fragment is a sequence of edges
$e_1e_2\ldots e_k$ such that $tar(e_i) = src(e_{i+1})$ for $1 \le i \le k-1$. A
\textit{cycle} is a path such that $tar(e_k) = src(e_1)$. 

A node $u$ is a \textit{descendant} of node $v$ if there exists a nonempty path
$e_1e_2\ldots e_k$ such that $src(e_1) = v$ and $tar(e_k)=u$. We also say that
$v$ is an \textit{ancestor} of $u$. Nodes are descendants/ancestors of
themselves. 

\begin{definition}[DAG and DAG Languages]\label{def:dag} A $\Sigma$-DAG
  (henceforth just referred to as a DAG) is a $\Sigma$-graph with no cycles (and
  a DAG fragment is a graph fragment with no cycles). Denote by $\calD_\Sigma$
  the set of all nonempty connected DAGs over $\Sigma$. A DAG language $L$ is a
  subset of $\calD_\Sigma$. Also denote by $\calH_\Sigma$ the set of nonempty
  connected DAG fragments over $\Sigma$. 
\end{definition}

\subsection{Automata Definition}
In this section we focus on the DAG automata studied
by~\cite{chiang2018weighted} and~\cite{blum2019properties}, but primarily use
the definitions and notation of the latter. The main difference is that the
latter enforces an ordering of the incoming and outgoing edges on each node, and
thus transitions are defined on sequences of states as opposed to multisets.
Further, Chiang's definition uses an unranked alphabet. These differences are
minor, as given an automaton as per Chiang's definition, we can represent it by
adding multiple transitions corresponding to each possible ordering of the state
multisets. The unranked alphabet can be converted to a doubly ranked alphabet by
taking all the transitions for each symbol $a$ and creating ranked symbol(s)
$a_{(h,t)}$ for each transition taking $h$ states to $t$ states.
\begin{definition}[DAG Automaton]\label{def:dagautomaton} A weighted DAG
  automaton is a tuple $M = \langle \Sigma, Q, \Theta, \delta, \mathbb{K}
  \rangle$ where:
  \begin{itemize}
    \item $\Sigma$ is a doubly ranked alphabet of node labels
    \item $Q$ is a finite set of states
    \item $\mathbb{K}$ is a commutative semiring
    \item $\Theta \subseteq Q^* \times \Sigma \times Q^*$ is a set of
    transitions, where each transition $\left( q_1q_2\ldots q_m, \sigma,
    p_1p_2\ldots p_n \right)$ is written $\alpha \xrightarrow{\sigma}\beta$,
    where $m,n \ge 0$.
    \item $\delta: Q^* \times \Sigma \times Q^* \rightarrow \mathbb{K}$ is a
    weight function assigning a weight to each transition. If $\delta(t) = w$,
    we also write the transition $t$ as $\alpha \xrightarrow{\sigma/w}\beta$.
    Note that for transitions not in $\Theta$, we define the weight as semiring
    $0$.
  \end{itemize}
\end{definition}

Transitions take a sequence of states and an alphabet symbol and output another
sequence of states. This corresponds to reading a DAG node (labelled with
$\sigma$) and associating states with the incoming and outgoing edge sequences
(using the implied sequence ordering). Further, we write transitions on root
nodes as $\lambda \xrightarrow{\sigma/w} \beta$, and transitions on leaf nodes
as $\alpha \xrightarrow{\sigma/w} \lambda$. 

We now define a \textbf{run} of automaton $M$ on DAG $D$ as a mapping from edges
to states, $\rho: E \rightarrow Q$. This lets us extend the weighting function
to runs, where we define the \textbf{run weight} of $\rho$ to be\footnote{When
we write $\rho(in(v))$ or $\rho(out(v))$, we mean the sequence $\rho(e_1)
\rho(e_2)\ldots \rho(e_k)$ where $in(v) = e_1e_2\ldots e_k$ is the sequence of
incoming edges}:

\[
\delta(\rho) = \bigotimes_{v \in V} \delta(\rho(in(v)), lab_V(v), \rho(out(v)))
\]

We also define the \textbf{graph weight} of $D$ under $M$, denoted by $[[M]](D)$
in \cite{chiang2018weighted}:
\[[[M]](D) = \bigoplus_{\text{all runs $\rho$ on $D$}} \delta(\rho)\]

Observe that unweighted DAG automata correspond to weighted DAG automata over
the Boolean semiring, and the DAG language \textbf{recognised} by
unweighted/Boolean semiring weighted DAG automaton $M$ is the set $L(M) = \{D
\in \calD_\Sigma \, | \, [[M]](D) = true\}$. The set of \textbf{recognisable DAG
languages} consists of languages for which there exists an unweighted DAG
automaton recognising it.

As pointed out in \cite{blum2019properties}, $L(M)$ only contains nonempty,
connected DAGs, since we define it over $\calD_\Sigma$. However, observe that
for $D_1, D_2 \in L(M)$, then the disconnected DAG written as $D = D_1 \& D_2 =
(V_1 \cup V_2, E_1 \cup E_2, lab_{V_1} \cup lab_{V_2}, in_1 \cup in_2, out_1
\cup out_2)$ is also accepted by $M$ assuming that the vertex and edge sets are
disjoint. Indeed, we can construct an accepting run $\rho_D = \rho_1 \cup
\rho_2$ given accepting runs $\rho_1,\rho_2$ for $D_1$ and $D_2$ respectively.
Hence, if $M$ accepts a language of nonempty, connected DAGs $L(M)$, it also
accepts the infinite language $L(A)^\&$, defined as the set of all DAGs of the
form $D_1 \& D_2 \& \ldots \& D_n$ for $D_i \in L(A)$ and $n \ge 0$ (where $n=$
corresponds to the empty DAG). We can think of this as similar to a Kleene
closure over the set of connected DAGs $L(M)$, although in this case $L(M)$ may
be infinite.  

\begin{observation}
Given disjoint $D_1, D_2 \in L(M)$, we have $D = D_1 \& D_2$ such that $[[M]](D)
= [[M]](D_1) \otimes [[M]](D_2)$
\end{observation}
\begin{proof}
  Each run of $M$ on $D$ can be decomposed into the union of a run on $D_1$ and
  a run on $D_2$, as the edge sets are disjoint. As such, if $\rho$ is a run on
  $D$, it follows from the definition of $\delta(\rho)$ and the disjointness of
  the vertex sets that $\delta(\rho) = \delta(\rho_1) \otimes \delta(\rho_2)$ where
  $\rho_1,\rho_2$ are runs on $D_1,D_2$ respectively. This gives us:

  \begin{align*}
    [[M]](D) &= \bigoplus_{\text{all }\{\rho_1,\rho_2\}} \delta(\rho_1) \otimes \delta(\rho_2)\\
    &= \bigoplus_{\text{all } \rho_1} \bigoplus_{\text{all } \rho_2} \delta(\rho_1) \otimes \delta(\rho_2)\\
    &= \bigoplus_{\text{all } \rho_1} \left(\delta(\rho_1) \otimes \bigoplus_{\text{all } \rho_2} \delta(\rho_2)\right) \\
    & = \left(\bigoplus_{\text{all } \rho_1} \delta(\rho_1)\right) \otimes \left(\bigoplus_{\text{all } \rho_2} \delta(\rho_2)\right) \\
    & = [[M]](D_1) \otimes [[M]](D_2)
  \end{align*}

\end{proof}

\section{Contexts, Partial Weights, and a Formal Correctness Proof of the Recognition Algorithm}
In this section, we define the notion of graph contexts and partial weights (a
similar concept to forward/backward weights of string automata). This was
inspired by the recognition/graph weight calculation algorithm presented by
\cite{chiang2018weighted} and we use these definitions to formalise and prove
correctness of the algorithm. 

\begin{definition}[Graph Context]\label{def:graphcontext}

  A graph context is a function $\cc: \calH_\Sigma^k \rightarrow \calH_\Sigma$,
  that takes in $k$ graph fragments and returns another graph fragment. Every
  graph context $\cc$ has a type $ type(\cc) = \langle (r_1,r_2,\ldots,r_k), r'
  \rangle$ such that the $i$'th argument fragment is of type $r_i$, and the
  returned graph fragment is of type $r'$. Further, a context has sequences of
  incoming and outgoing edges $E_h = e_1e_2\ldots e_h, E_t = e'_1e'_2\ldots
  e'_t$ for $r' = (h,t)$. 

  We define the set of graph contexts $C_\Sigma$ over doubly ranked alphabet
  $\Sigma$ as the smallest set defined inductively according to the following:

  \begin{itemize}
    \item Consider the function $\cc$ of type $\langle (r),r\rangle$ such that
    $\cc[\dd_r] = \dd_r$ for any $r$-graph fragment. Then $\cc \in C_\Sigma$,
    and $E_h,E_t$ are the same as those of $\dd_r$. These are `trivial' contexts
    and can be thought of as a placeholder/gap.

    \item Let $\dd_r$ be a $r$-graph fragment consisting of $k$ nodes, and let
    $v_1,v_2,\ldots, v_k$ be a sequence of these nodes whose labels have ranks
    $r'_1, r'_2,\ldots, r'_k$ respectively. Let $\cc_i \in C_\Sigma$ be a
    context of type $\langle (r''_{1_i},\ldots,r''_{m_i}), r'_i\rangle$. Then,
    let $\cc$ be the function obtained by replacing the nodes $v_1,v_2,\ldots
    v_k$ in $\dd_r$ with contexts $\cc_1, \cc_2,\ldots,\cc_k$ (matching the
    edges based on the ordering). Then, $\cc \in C_\Sigma$ and $\cc$ is a
    context with rank $\langle
    (r''_{1_1},\ldots,r''_{m_1},\ldots,r''_{1_k},\ldots,r''_{m_k}),r\rangle$.
    The incoming and outgoing edge sequences $E_h, E_t$ are the same as those of
    $\dd_r$, and some of these edges may also be identified with edges of an
    internal context $\cc_i$. Informally speaking, taking a graph fragment and
    replacing all vertices with contexts generates a new context. 
  \end{itemize}

  Let $\cc$ be a context of type $\langle (r_1,r_2,\ldots,r_k), r' \rangle$.
  Given graph fragments $\dd_{r_1}, \dd_{r_2}, \ldots, \dd_{r_k}$ we refer to
  the term $\cc[\dd_{r_1}, \dd_{r_2}, \ldots, \dd_{r_k}]$ as a \emph{grounded
  context}, and such a term is a $r'$-graph fragment. When performing this
  substitution, each `argument' of the context $\cc$ is replaced with a graph
  fragment of matching rank, and the incoming/outgoing edges are identified with
  each other based on the ordering. 

\end{definition}

This definition also implies that contexts can be nested (provided the types
agree). For instance, a context $\cc$ with type $\langle (r, r'), r\rangle$ can
be nested into its first argument to form $\cc' = \cc[\cc, \cc_{r'}]$, which is
a $\langle (r,r',r'), r \rangle$ context (where we use the trivial context
$\cc_{r'}$ of type $\langle (r'), r' \rangle$ for completeness). 

Furthermore, observe that this construction is similar to a grammar-like
derivation of DAGs, starting with a trivial context of type $\langle (r), r
\rangle$ where $r = (0,0)$. Graphs could then be generated with rules that
replace contexts with fragments/other contexts, with fragments representing
terminals and contexts representing nonterminals. This is similar to node
replacement graph grammars and relationships between such grammars and automata
may be worth looking into.

\begin{lemma}\label{lem:onegroundedcontext} Every graph fragment can be
  expressed as at least one grounded context.
\end{lemma}

\begin{proof}
  Let $\dd_r$ be an $r$-graph fragment with $n$ nodes. Given some ordering of
  the nodes $1,\ldots,n$, if we replace every node $i$ of rank $r_i$ with a
  trivial context of type $\langle (r_i),r_i \rangle$, we obtain a context $\cc$
  with arguments corresponding to each node of $\dd_r$ and an output type of
  $r$, as per the inductive definition of a graph context. Let $\dd^{(i)}_{r_i}$
  be the $r_i$-graph fragment containing just node $i$ with label rank $r_i$.
  Then, since the arguments of $\cc$ are in the node order of $1,\ldots,n$, it
  is easy to see that $\cc[\dd^{(1)}_{r_1}, \dd^{(2)}_{r_2},\ldots,
  \dd^{(n)}_{r_n}] = \dd_r$.
\end{proof}

For graph fragments with more than 1 node, there are multiple ways to express
the fragment as a grounded context.

\begin{definition}[Partial Weight]\label{def:partialweight} Let $M = \langle
  \Sigma, Q, \Theta, \delta, \K \rangle$ be a DAG automaton, and let $\dd_r$ be
  a graph fragment with $r = (h,t)$. We define the \textbf{partial weight}
  $\aalpha$ of the graph fragment given a mapping $f: E_h \cup E_t \rightarrow
  Q$ inductively as follows:
  \begin{itemize}
    \item For a graph fragment of 1 node $v$, we have 
    \[
    \aalpha[\dd_r, f] = \delta(f(E_h), lab_V(v), f(E_t))
    \]

    \item For a graph fragment with $n > 1$ nodes, by the construction in Lemma
    \ref{lem:onegroundedcontext}, we can express $\dd_r = \cc[\dd^{(1)}_{r_1},
    \dd^{(2)}_{r_2},\ldots, \dd^{(n)}_{r_n}]$. We then define:
    \[
    \aalpha[\dd_r,f] = \aalpha[\cc[\dd^{(1)}_{r_1}, \dd^{(2)}_{r_2},\ldots,
    \dd^{(n)}_{r_n}],f] =  \bigoplus_{\text{all }\calF(f)} \bigotimes_{i=1}^n \aalpha[\dd^{(i)}_{r_i}, f_i]
    \]
    Where $\calF(f)$ is an (ordered) set of mappings $\{f_1,f_2,\ldots,f_n\}$
    such that\footnote{If $r = (x,y)$ we say $r[0] = x, r[1]=y$} $f_i:
    E_{r_i[0]} \cup E_{r_i[1]} \rightarrow Q$, which maps the `dangling' edges
    of $\dd^{(i)}_{r_i}$ to states. We require that the set of these mappings
    has no conflicts, both amongst the $f_i$'s and with $f$, such that if we
    were to draw the full graph fragment $\dd_r$ and assign states to edges
    based on the functions in $\calF(f)$, each edge would have exactly one
    state. More formally, we must obtain a function $\rho: E \cup E_h \cup E_t
    \rightarrow Q$ for $\dd_r = (V,E,lab_V,in,out,E_h,E_t)$ such that $\rho =
    \bigcup_i \calF(f)_i$ and $\rho$ agrees with $f$ on $E_h\cup E_t$.
    
    \item We also extend the above definition for the case where $\dd_r$ is
    expressed as a different grounded context, i.e $\dd_r = \cc[\tt_1,
    \tt_2,\ldots,\tt_k]$, where $\tt_i$ is some grounded context:
    \[
    \aalpha[\cc[\tt_1,\tt_2,\ldots,\tt_k],f] =  \bigoplus_{\text{all }\calF(f)} \bigotimes_{i=1}^k \aalpha[\tt_i, \calF(f)_i]
    \]
  \end{itemize}
  
\end{definition}

We now show that the above definition is consistent, i.e two identical graph
fragments expressed as different grounded contexts will have the same partial
weight, and that it aligns with the definition of graph weight. 


\begin{lemma}\label{lem:partialweightequivalence} Let $\dd_r$ be an $r$-graph
  fragment with $n>1$ nodes. We have from Lemma \ref{lem:onegroundedcontext}
  that $\dd_r = \cc_1[\dd^{(1)}_{r_1}, \dd^{(2)}_{r_2},\ldots,
  \dd^{(n)}_{r_n}]$. Let $ \dd_r = \cc_2[\tt_1, \tt_2, \ldots, \tt_k]$ be
  another grounded context equivalent to $\dd_r$, where $\tt_i$ is a grounded
  context. We have that:

  \[
    \aalpha[\cc_1[\dd^{(1)}_{r_1}, \dd^{(2)}_{r_2},\ldots,
    \dd^{(n)}_{r_n}],f] = \aalpha[\cc_2[\tt_1,\tt_2,\ldots,\tt_k],f]
  \]
\end{lemma}

\begin{proof}

  Each $\tt_i$, being a grounded context, is itself a graph fragment and can
  therefore be expressed as $\cc_i[\dd^{(i,1)}_{r_{i,1}},
  \dd^{(i,2)}_{r_{i,2}},\ldots, \dd^{(i,m_i)}_{r_{i,m_i}}]$ as in
  Lemma~\ref{lem:onegroundedcontext} such that $\dd^{(i,1)}_{r_{i,1}}$ is a
  single node. Then, since the $\tt_i$'s form a partition of $\dd_r$'s nodes,
  the union of all the nodes of the $\tt_i$'s is exactly the set of nodes in
  $\dd_r$, i.e $\bigcup_{i=1}^k \bigcup_{j=1}^{m_i} \{\dd^{(i,j)}_{r_{i,j}}\} =
  \bigcup_{i=1}^n \{\dd^{(i)}_{r_i}\}$. 

  We have that:
  
  \begin{align*}
    \aalpha[\cc_2[\tt_1,\ldots,\tt_k],f] &= \bigoplus_{\text{all } \calF(f)} \bigotimes_{i=1}^k \aalpha[\tt_i, f_i]\\
    &= \bigoplus_{\text{all } \calF(f)} \bigotimes_{i=1}^k \left(\bigoplus_{\text{all } \calF'(f_i)} \bigotimes_{j=1}^{m_i} \aalpha[\dd^{(i,j)}_{r_{i,j}}, f'_{i,j}] \right)\\
  \end{align*}

  Note that $f_i$ is a mapping of dangling edges of $\tt_i$ to states in $Q$,
  and $\calF'(f_i)$ is another set of nonconflicting mappings $\{f'_{i,1},
  \ldots, f'_{i,m_i}\}$ of which $f'_{i,j}$ maps dangling edges of node
  $\dd^{(i,j)}_{r_{i,j}}$ to states in $Q$. 

  We can then express the above equation as:
  \begin{align*}
    \aalpha[\cc_2[\tt_1,\ldots,\tt_k],f] &= \bigoplus_{\text{all } \calF(f)} \bigotimes_{i=1}^k \left(\bigoplus_{\text{all } \calF'(f_i)} \bigotimes_{j=1}^{m_i} \aalpha[\dd^{(j_i)}_{r_{j_i}}, f'_{i,j}] \right)\\
    &= \bigoplus_{\text{all } \calF(f)} \bigoplus_{\text{all } \mathcal{F'}(f_1),\ldots, \mathcal{F'}(f_k) } \bigotimes_{i=1}^k \bigotimes_{j=1}^{m_i} \aalpha[\dd^{(j_i)}_{r_{j_i}}, f'_{i,j}] \\
    &= \bigoplus_{\text{all } \calF^*(f)}\bigotimes_{i=1}^n \aalpha[\dd^{(i)}_{r_{i}}, f^*_{i}] \\
    &= \aalpha[\cc_1[\dd^{(1)}_{r_1}, \dd^{(2)}_{r_2},\ldots,
    \dd^{(n)}_{r_n}],f] 
  \end{align*}

  Where given some $\calF(f)$, we define $\calF^*(f) = \bigcup_{i=1}^k
  \calF'(f_i)  =  \bigcup_{i=1}^k \bigcup_{j=1}^{m_i} \{f'_{i,j}\} = \{f^*_1,
  f^*_2, \ldots, f^*_n\} $ as from above we know that the $\tt_i$'s partition
  the nodes of $\dd_r$ (and assuming some consistent ordering). 
\end{proof}

\begin{lemma}\label{lem:partialweightgraphweight} Let $M$ be a DAG automaton.
  For a graph fragment $\dd_r$ with $r = (0,0)$, i.e a standard DAG, we have:
  \[
  \aalpha[\dd_r, \emptyset] = [[M]](\dd_r) 
  \]
  
\end{lemma}
Note: since a $(0,0)$ ranked graph fragment has no dangling edges, we define the
only possible mapping from its dangling edges to states as $\emptyset$. This
also means that $\calF(\emptyset)$ need only be internally consistent amongst
the $f_i$'s.

\begin{proof}
Let $n$ be the number of nodes in $\dd_r$. Since $\calF(\emptyset) = \{f_1,
f_2,\ldots f_n\}$ corresponds to a set of non-conflicting mappings from edges to
states, taking the union of all the mappings gives us a run $\rho: E \rightarrow
Q$, where $E$ is the set of edges of $\dd_r$. Again using the construction from
Lemma 4.3 to get $\dd_r = \cc[\dd^{(1)}_{r_1}, \dd^{(2)}_{r_2},\ldots,
\dd^{(n)}_{r_n}]$ we get:

\begin{align*}
  \aalpha[\dd_r, \emptyset] &= \bigoplus_{\text{all }\calF(f)} \bigotimes_{i=1}^n \aalpha[\dd^{(i)}_{r_i}, \calF(f)_i]\\
  &= \bigoplus_{\text{all } \rho} \bigotimes_{i=1}^n \delta(\rho(E_h), lab_V(v), \rho(E_t)) &&\quad \dd^{(i)}_{r_i}\text{ has only 1 node } v\\
  &=  \bigoplus_{\text{all } \rho} \delta(\rho)\\
  &= [[M]](\dd_r)
\end{align*}
\end{proof}

\begin{theorem}
  The recognition/graph weight computation algorithm provided in
  \cite{chiang2018weighted} (and given in the appendix) is correct, i.e we
  return $[[M]](D)$ for input DAG $D$. 
\end{theorem}
\begin{proof}
  Let $\dd$ be the input DAG of rank $r = (0,0)$ with $n$ nodes. 

  We first show that after each edge contraction, the algorithm maintains a
  grounded context decomposition $\dd = \cc[\tt_1, \ldots, \tt_k]$ as well as
  maintaining the values of $\aalpha[\tt_i, f]$ for all $\tt_i$ and all possible
  $f$. We show this by induction on the number of edge contractions.

  Before the first edge contraction, we have $\dd = \cc_1[\dd^{(1)}_{r_1},
  \dd^{(2)}_{r_2},\ldots, \dd^{(n)}_{r_n}]$, and the first section of the
  algorithm iterates through all these nodes and computes all
  $\aalpha[\dd^{(i)}_{r_i}, f]$. 

  For the inductive case, assume that after $m-1$ edge contractions, we have a
  decomposition $\dd = \cc[\tt_1, \ldots, \tt_k]$ and values of all
  $\aalpha[\tt_i, f]$. When we contract the next edge, observe that by design we
  contract an edge between two graph fragments $\tt_i$ and $\tt_j$ since we
  delete any intra-fragment edges from the contraction candidates (edges in
  $I$). By contracting this edge, the algorithm forms a `supervertex' $v$
  corresponding to a larger graph fragment $\tt' = \cc'[\tt_i, \tt_j]$ for some
  context $\cc'$ (implicitly defined). We then compute the `dangling edges',
  denoted by $star(v)$ in the algorithm, by combining the dangling edges of
  $\tt_i$ and $\tt_j$ (excluding edges between the two fragments). 

  Now, by definition for a mapping $h: star(v) \rightarrow Q$, if we have
  mappings $f,g$ for the dangling edges of $\tt_i,\tt_j$ respectively such that
  $f,g$ do not conflict on shared edges ($f|_I = g|_I$) and $h = f \cup g
  \setminus f|_I$, the set $\{f,g\}$ is a valid $\calF(h)$. Moreover, any
  $\calF(h)$ must be expressible in this form. Using the definition of partial
  weight and Lemma \ref{lem:partialweightequivalence} we get:

  \begin{align*}
    \aalpha[\cc'[\tt_i, \tt_j],h] &= \bigoplus_{\text{all }\calF(h)} \aalpha[\tt_i, h_i] \otimes \aalpha[\tt_j, h_j]\\
    &= \bigoplus_{\text{all }h = f \cup g \setminus f|_I} \aalpha[\tt_i, f] \otimes \aalpha[\tt_j, g]
  \end{align*}
  This is exactly what the recognition algorithm computes. 

  Finally, since in each edge contraction we merge two graph fragments, after
  $n-1$ edge contractions we terminate at a decomposition $\dd =
  \cc''[\tt_{final}]$ and compute $\aalpha[\cc''[\tt],h]$, where $h = \emptyset$
  as $\dd$ has rank $(0,0)$. By Lemma \ref{lem:partialweightgraphweight}, this
  is equal to $[[M]](\dd)$. 
\end{proof}

\section{Determinism, Hypergraph Representation, and Equivalence of States}

\cite{blum2019properties} define notions of equivalence and minimisation for
top-down deterministic DAG automata, and in doing so prove that the equivalence
problem is decidable in polynomial time for deterministic automata. In this
section we explore these ideas and present alternative but equivalent
definitions for equivalence of states by viewing a DAG automaton as a
hypergraph. In this sense, we mirrors the definitions of state equivalence for
tree automata. 

\begin{definition}[Deterministic Automaton]\label{def:deterministicautomaton} A
  DAG automaton $M =  \langle \Sigma, Q, \Theta, \delta, \mathbb{K}\rangle$ is
  top down deterministic if given a symbol $\sigma \in \Sigma$, for each $\alpha
  \in Q^*$ we have at most one $\beta \in Q^*$ such that $ \alpha
  \xrightarrow{\sigma} \beta$ is a rule in $\Theta$. An automaton is bottom up
  deterministic if for a symbol $\sigma \in \Sigma$, for each $\beta \in Q^*$,
  we have at most one $\alpha \in Q^*$ such that $\alpha \xrightarrow{\sigma}
  \beta$ is a rule in $\Theta$.

\end{definition}

In a deterministic automaton, there is at most one accepting run $\rho$ per DAG
$D$ (but this does not necessarily hold for DAG fragments).

It the class of deterministic recognisable DAG languages, i.e the DAG languages
recognised by either a bottom up or top down automaton, is a strict subset of
the class of recognisable DAG languages. In other words, determinism strictly
decreases the power of our automaton. The argument, presented in
\cite{blum2019properties}, is quite simple and we summarise the main ideas.
Regular tree languages are also regular DAG languages as DAG automata generalise
tree automata. Now consider a top down deterministic DAG automaton recognising a
tree language (which is also a DAG language). Such a DAG automaton can also be
expressed as a top down deterministic tree automaton. However, since there are
regular tree languages that cannot be recognised by a top down deterministic
tree automaton, such a language (which is also a regular DAG language) also
cannot be recognised by a top down deterministic DAG automaton.   

For the remainder of this section we will focus on top down deterministic
automata. Note however that the concepts still apply to bottom up deterministic
automata, as we can view a bottom up deterministic automaton as a top down one
accepting the `reverse' language (i.e the DAG edge directions are reversed).
This also implies bottom up deterministic DAG automata have the same power as
top down deterministic automata, and are strictly weaker than nondeterministic
automata. This notion of a `dual' automaton is delineated in
\cite{blum2019properties}. 

\begin{definition}[Reduced Automaton]\label{def:reducedautomaton}

  A DAG automaton $M =  \langle \Sigma, Q, \Theta, \delta, \mathbb{K}\rangle$ is
  reduced if there are no useless transitions. A transition $t \in \Theta$ is
  not useless if there is at least one DAG $D \in L(M)$ for which there exists a
  run $\rho$ with $\delta(\rho) \ne 0$, such that $t = (\rho(in(v)), lab_V(v),
  \rho(out(v)))$ for some vertex $v$ in $G$. 
\end{definition}

For the remainder of this section we will assume DAGs are reduced. This can be
performed in polynomial time as shown in \cite{blum2019properties}.

\begin{definition}[Automata Hypergraphs and
Hyperpaths]\label{def:automatahyperpaths}
  
  Let $M =  \langle \Sigma, Q, \Theta, \delta, \mathbb{K}\rangle$ be a DAG
  Automaton. We can view this automaton as a hypergraph, with each state $q \in
  Q$ forming a node, and each transition $\alpha\overset{a}{\rightarrow}\beta$
  representing a hyperedge $e$ with $src(e) = \alpha$ and $tar(e) = \beta$,
  where $\alpha,\beta \in Q^*$. Note that transitions are between ordered
  sequences of states, and therefore we may have two hyperedges/transitions
  between two sets of states, with the only difference being the orderings. 

  
  Then, we define a \textbf{hyperpath} $\ppi$ through this hypergraph as a tuple
  $(I,F,S)$, where $I$ is an initial ordered sequence of states $q_1\ldots q_m$,
  $F$ is a final ordered sequence of states $p_1\ldots p_n$, and $S$ is a
  sequence of hyperedges/transitions $t_1,t_2,\ldots,t_k$ such that the
  following algorithm takes in $I,S$ as input and returns a multiset with the
  same elements as $F$:

  \begin{algorithm*}[H]
    \caption{$\textsc{HyperpathTraversal}(I,S)$}
	  \label{alg:hyperpathTraversal}
    $states \gets \{I\}$ \Comment*[r]{$\{I\}$ is the multiset of sequence
    elements} \ForEach{$\alpha \xrightarrow{a} \beta  \in S$
    \Comment*[r]{Iterate in sequence order}}{ $states \gets (states \setminus
    \{\alpha\}) \cup \{\beta\}$\\
    } \Return{states}
  \end{algorithm*}
  
  This effectively enforces a topological order on the hyperedge sequence such
  that we can only pick a hyperedge in our path if its source states are
  `available'.

  We define the \textbf{weight} of a hyperpath $\ppi = (I,F,S)$ as $\ww(\ppi) =
  \bigotimes_{t \in S} \delta(t)$. 
\end{definition}

\begin{definition}[Yield of Hyperpath]\label{def:hyperpathyield}

  The unique \textbf{yield} of a hyperpath $(I,F,S)$ is the graph fragment $\dd
  = (V,E,src,tar,lab_V,E_h,E_t)$ and run $\rho: E \cup E_h \cup E_t \rightarrow
  Q$ such that:
  \begin{itemize}
    \item $\rho(E_h) = I$, i.e each `head' edge is assigned the corresponding
    state in the initial sequence (we match the $i$'th edge with the $i$'th
    state)
    \item $\rho(E_t) = F$
    \item For every $v \in V$, $(\rho(in(v)), lab_V(v), \rho(out(v)) )\in S$.
  \end{itemize}
\end{definition}


With this definition, we can see that a run $\rho$ of automaton $M$ on a DAG $D$
corresponds to a hyperpath $\ppi$ in $M$'s hypergraph with that $I = F =
\lambda$ such that $\ppi$ yields $G$ and $\ww(\ppi) = \delta(\rho)$. 

We now have the tools to present an alternative characterisation of equivalent
states in an unweighted DAG automaton. This mirrors the state equivalence
definition of tree automata, using the hypergraph representation of the
automaton. We also show that this definition is equivalent to the inductive
definition of \cite{blum2019properties}.

\begin{definition}[Equivalence and Distinguishability of States
(Unweighted)]\label{def:unweightedequivalence}
  
  Let $M =  \langle \Sigma, Q, \Theta\rangle$ be a reduced top down
  deterministic DAG automaton (unweighted so we omit $\delta$ and $\K$).
  
  Two states $q_1, q_2 \in Q$ are \textbf{equivalent} if the following holds:
  For all sequences $I_1$ with an occurrence of $q_1$ at position $i$, if there
  exists a hyperpath from $I_1$ to some $F_1$ yielding $\dd$ with run $\rho_1$,
  then if we replace position $i$ in $I_1$ with $q_2$ to form $I_2$, then there
  must exist a hyperpath from $I_2$ to some $F_2$ yielding $\dd$ with run
  $\rho_2$ obeying the following:
  
  \[
    \rho_2(e) = \begin{cases}
      \rho_1(e) &\quad \text{ for } e \in E_h \text{ except the $i$'th edge } e_i\\
      q_2 &\quad \text{ for the $i$'th edge in $E_h$, } e_i\\
    \end{cases}
  \]
  For all other $e \in E \cup E_t$ we place no restrictions on $\rho_2$. 

  If the above doesn't hold, then $q_1$ and $q_2$ are \textbf{distinguishable}.

\end{definition}


\begin{theorem}
  This definition of distinguishable states is equivalent to that of
  \cite{blum2019properties} (provided in Appendix).
\end{theorem}
\begin{proof}
  The base case of Blum's definition says that two states $p,p'$ are
  distinguishable if there exists a transition $\alpha p \alpha'
  \overset{a}{\rightarrow} \beta$ but no transition $\alpha p' \alpha'
  \overset{a}{\rightarrow} \beta'$ for $\alpha,\alpha',\beta,\beta' \in Q^*$. In
  this case, let $I_1 = \alpha p \alpha'$, $F_1 = \beta$. Then, there exists the
  hyperpath containing just the hyperedge/transition $\alpha p \alpha'
  \overset{a}{\rightarrow} \beta$ yielding $a$ but there does not exist a
  hyperpath yielding $a$ for $I' = \alpha p'\alpha'$ since no single transition
  consuming symbol $a$ has source states $\alpha p'\alpha'$.

  For the inductive case, in Blum's definition, $p,p'$ are distinguishable if
  there exist transitions $\alpha_0 p \alpha_0' \overset{a_0}{\rightarrow}
  \beta_0 q_0 \beta_0'$ and  $\alpha_0 p' \alpha_0' \overset{a_0}{\rightarrow}
  \gamma_0 q_0' \gamma_0'$ such that $q_0$ and $q_0'$ are distinguishable.
  `Unrolling' this definition, we must have some sequence of transitions
  $\alpha_0 p \alpha_0' \overset{a_0}{\rightarrow} \beta_0 q_0 \beta_0',
  \alpha_1 q_0 \alpha_1' \overset{a_1}{\rightarrow} \beta_1 q_1 \beta_1',\ldots,
  \alpha_k q_{k-1} \alpha_k' \overset{a_k}{\rightarrow} \beta_k$ and $\alpha_0
  p' \alpha_0' \overset{a_0}{\rightarrow} \gamma_0 q_0' \gamma_0', \alpha_1 q_0'
  \alpha_1' \overset{a_1}{\rightarrow} \gamma_1 q_1' \gamma_1',\ldots,
  \alpha_{k-1} q_{k-2}' \alpha_{k-1}' \overset{a_{k-1}}{\rightarrow}
  \gamma_{k-1} q_{k-1}' \gamma_{k-1}'$ such that there is no transition
  $\alpha_k q_{k-1}' \alpha_k' \overset{a_k}{\rightarrow} ...$

  As such, if we let $I =  \alpha_0 p \alpha_0' \alpha_1 \alpha_1'\ldots
  \alpha_k\alpha_k'$ and $F =
  \beta_0\beta_0'\beta_1\beta_1'\ldots\beta_{k-1}\beta_{k-1}' \beta_k$ the
  hyperpath formed by the first sequence of rules above yields the graph
  fragment $\dd$, consisting of a chain of $a_0, a_1,\ldots, a_k$ (and many
  dangling edges). We also get the run $\rho_1$ which assigns states to edges
  based on the transitions in the hyperpath. However, if we replace $p$ by $p'$
  in $I$ to form $I'$, we have no hyperpath yielding $\dd$ with run $\rho_2$ as
  defined in Def \ref{def:unweightedequivalence}. To see why, observe that as
  per the definition of $\rho_2$ and top down determinism, we must take the
  transition $\alpha_0 p' \alpha_0' \xrightarrow{a_0} \gamma_0 q_0' \gamma_0'$
  in order to yield the first node $a_0$ with the same ordered sequence of
  edges. Similarly, we must then yield a node with label $a_1$ such that
  incoming edges from $E_h$ must match the $a_1$ node in $\dd$ and their states
  in $\rho_2$ must match $\rho_1$. Further, the $j$'th outgoing edge from node
  $a_0$ (assigned state $q_0$) is connected to an incoming edge of node $a_1$.
  This is only possible if we take the transition $\alpha_1 q_0' \alpha_1'
  \xrightarrow{a_1} \gamma_1 q_1' \gamma_1'$. By repeating this argument, we
  conclude that we must take transitions as per the second sequence above, until
  we reach the last transition, at which point we cannot yield a node labelled
  $a_k$ since we have no transition $\alpha_k q_{k-1}' \alpha_k'
  \overset{a_k}{\rightarrow} ...$.

  We have thus shown that any two states distinguished according to Blum's
  definition are distinguished by our definition. 
  
  We now show the converse, i.e any states distinguished by our definition are
  also distinguished by that of Blum. Let $q, q'$ be two distinguishable states
  such that there exists $I,F$ with a hyperpath $\ppi$ between them yielding
  $\dd$ and $q$ occurs in the $i$'th position of $I$. As these states are
  distinguishable, if we replace the $i$'th state in $I$ with $q'$ to create
  $I'$, there is no hyperpath from $I'$ to some $F'$ yielding $\dd$. Let $S =
  t_1t_2\ldots t_k$ be the hyperpath yielding $\dd$ prior to state replacement.
  We shall induct on the length of $S$.

  The base case where $k=1$ is simple, as the existence of a graph fragment
  $\dd$ implies there is a rule $\alpha q \alpha' \xrightarrow{a} \beta$, but
  the nonexistence of a graph fragment containing the node $a$ after state
  replacement implies there is no rule $\alpha q' \alpha' \xrightarrow{a}
  \ldots$, aligning with the base case of Blum's definition.

  We will split the inductive case into two cases. Consider the first transition
  $t_1 = \alpha \xrightarrow{a} \beta$. If the $i$'th state $q$ is not a part of
  $\alpha$, then replacing $q$ with $q'$ makes no difference to this transition.
  Therefore, we can remove this transition and create $I_2$ by removing $\alpha$
  from $I$ and appending $\beta$. This results in a new DAG fragment $\dd'$
  yielded by hyperpath $S' = t_2\ldots t_k$, such that $\dd'$ cannot be yielded
  by replacing $q$ with $q'$ in $I_2$, and we apply the inductive hypothesis. 

  For the other case, we have $t_1 = \alpha q \alpha' \xrightarrow{a} \beta$. If
  we cannot construct $\dd$ post state replacement because there is no rule
  $\alpha q' \alpha' \xrightarrow{a} \ldots$ then we are done immediately. If on
  the other hand there is a rule $\alpha q' \alpha' \xrightarrow{a} \gamma$,
  then the state replacement has no effect on this node with label $a$.
  Therefore, we can create $I_2$ by removing  $\alpha q \alpha'$ from $I$ and
  appending $\beta$. Then, $S' = t_2 \ldots t_k$ must yield a graph fragment
  $\dd'$ similar to $\dd$ but without the `first' node labelled $a$. However,
  there must be a state $p \in \beta$ and a state $p' \in \gamma$ such that
  replacing $p$ with $p'$ then there cannot be a hyperpath yielding $\dd'$ (if
  there was, we contradict our initial assumption that replacing $q$ with $q'$
  makes it impossible to yield $\dd$). Thus, we can apply the inductive
  hypothesis and therefore the statement holds for all hyperpath lengths.

  We have now shown that if two states are distinguished by our definition, they
  are also distinguished by that of Blum's, and combining with the first part,
  we conclude that they are equivalent. 
\end{proof}

We also present a definition of cyclic automata.

\begin{definition}
  (Cyclic Automaton)

  A reduced DAG automaton $M =  \langle \Sigma, Q, \Theta \rangle$ is cyclic if
  there exists a hyperpath $ \ppi = (K,K,S)$, $K \ne \lambda$, where $S$ is a nonempty sequence
  of transitions such that $\ppi$ yields a connected DAG fragment. In other
  words, there is a sequence of transitions that takes an initial sequence of
  states to the same sequence of states. This can also be viewed as the
  hypergraph of the automaton $M$ being cyclic. 
\end{definition}

\begin{conjecture}
  Given a reduced DAG automaton $M = \langle \Sigma, Q, \Theta \rangle$, the
  language of connected DAGs recognised by $M$, denoted $L(M)$ is infinite iff
  $M$ is cyclic.
\end{conjecture}

\begin{proof}[Proof Idea]

  If $L(M)$ is infinite, then for some $D \in L(M)$ and a nonzero
  weight run $\rho$ on $D$, we must have a sequence of connected transitions
  $t_1,t_2\ldots,t_k,t_1$. By connected transitions we mean that two consecutive
  transitions share an edge with the same state under $\rho$. This follows as
  $L(M)$ being infinite means that path lengths in the DAGs are unbounded, but
  the set of transitions is finite. Let $e_1,e_2,\ldots,e_k$ be the path
  that connects the transitions, i.e $t_1$ and $t_2$ both use edge $e_1$, $t_2$
  and $t_3$ both use edge $e_2$, and $t_k$ and the repeat $t_1$ both use $e_k$.

  We claim that for $t_i = \alpha_i \xrightarrow{a_i} \beta_i$, if we set $K =
  \alpha_1 \alpha_2 \ldots \alpha_k \setminus \{\rho(e_1),\rho(e_2),
  \ldots,\rho(e_k)\}$ (meaning we remove all instances of state $\rho(e_i)$), we
  can find an $S$ that takes $K$ to $K$, i.e we can find a hyperpath $\ppi =
  (K,K,S)$. 

  We leave this as a conjecture, and defer the proof to a future paper but the
  idea is to use the hyperpath that yields $D$ (which includes the sequence of
  transitions $t_1,\ldots, t_k$) and `rearrange' the connections. See the
  following diagram for an example:
  % https://tikzcd.yichuanshen.de/#N4Igdg9gJgpgziAXAbVABwnAlgFyxMJZABgBoBGAXVJADcBDAGwFcYkQBHAfXJAF9S6TLnyEU5UgCZqdJq3YAjfoJAZseAkQlUaDFm0Scuk5UPWitpAMwy98w9xMCzIzeNLFbcgyHqnVwhpiyJLWXvrs3Fb+aq7BZNK63uwAxjGBFihkNkkRDjzp5m7IEgAs4fYgSs4BRcGh5bmVUIVxlgCsFT6OrUFEoZ1N3VzRNbF97gBsXexsYxnFpR4z+bzzdUTty0ORxr2ZyEvTO6v7xZOkjbJ5Rk4q4wdbOteV1fcLwQDsUiu+Z8FbRIvYZ3FwTQ4UX5pdZtFBLIF2YZrd4bFBbHLAxT-IgXBHJfKjFGw5BbY6Y-Kg2rE75XRG7Qlgg4XDF0wwtGHgi6DckgOYyGBQADm8CIoAAZgAnCAAWyQZBAOAgSAk5LAzEYjBojHoChgjAACh92IwYGKcP5JTKkKEFUrEOQapbZfaaIqkFZHVLnUtbXLPVbED63YhJP7nSrg+0w0gLr7EJ9o-HXXaAByJqzJpBplROpBbOMATkTKcz9uIxdL5AdOa9yvlwfIoZrAarlab4tr9ozcfICeb4aDdt7ifIJZ7Uf7yu+PaLk5D9btkmrHYDS9LknbIFz9oL683297649c4369KidPcckE5X4bHDcmF+7wckj7n5FjL77t+t+a-fEoPggA
  \begin{center}
    \begin{tikzcd}
      & a \arrow[ld] \arrow[d] &                &  & q_1 \arrow[rd] & q_2 \arrow[d]             &               &                               \\
    q_1 \arrow[rd] & q_2 \arrow[d]          &                &  & c \arrow[d]    & b \arrow[d] \arrow[rd]    &               &                               \\
    c \arrow[d]    & b \arrow[d] \arrow[rd] &                &  & q_1 \arrow[rd] & q_2 \arrow[d]             & q_3 \arrow[d] & a \arrow[llldddd] \arrow[ldd] \\
    q_1 \arrow[rd] & q_2 \arrow[d]          & q_3 \arrow[d]  &  &                & b \arrow[ddd] \arrow[rrd] & d             &                               \\
      & b \arrow[d] \arrow[rd] & d              &  &                &                           & q_2 \arrow[d] & q_3 \arrow[ld]                \\
      & q_2 \arrow[d]          & q_3 \arrow[ld] &  &                &                           & e             &                               \\
      & e                      &                &  & q_1            & q_2                       &               &                              
    \end{tikzcd}
  \end{center}
  Where the left diagram is $D$, with a sequence of transitions $q_1q_2
  \xrightarrow{b} q_2q_3, \lambda \xrightarrow{c} q_1, q_1q_2 \xrightarrow{b}
  q_2q_3$, and the right diagram represents the yield of a hyperpath from $K =
  q_1q_2$ to itself again. Observe that the transitions used are exactly the
  same as those of $D$ but just rearranged. 

  Similarly if $M$ is cyclic, then there exists a hyperpath $\ppi = (K,K,S)$ yielding a
  connected DAG fragment $\dd$. We can then `pump' this by traversing the
  hyperpath multiple times, as we always reach the same set of states. As $\dd$
  is connected, by repeatedly traversing/pumping the hyperpath $k$ times, we
  yield a larger connected DAG fragment which we shall denote $\dd^k$. However,
  this process must also have a way to terminate, i.e yield a rank $(0,0)$ DAG
  fragment. 

  We claim that this termination process to `tie up' the dangling edges can be
  done in a manner similar to the above diagram. Since $M$ is reduced, each
  transition $\tt_i$ is used in some DAG $\dd_i$ for some nonzero weight run
  $\rho_i$. We can thus construct each of these DAGs $\dd_i$ and remove the node
  corresponding to the transition $\tt_i$, leaving us with dangling edges
  corresponding to the left and right hand side states of $\tt_i$. Then, we
  claim that the set of these (currently disconnected) DAGs $\dd_i$ can be
  connected with the DAG fragment $\dd$ to yield an overall connected DAG.
  Hence, we can pump this infinitely to yield infinitely many connected DAGs. 

  Again, we leave this termination process as a conjecture to be formally proved in a future paper. 

\end{proof}

We believe this definition of cyclic automata is equivalent to that of
\cite{blum2019properties}, as assuming the above conjectures hold, we obtain the
same result that $M$ is cyclic $\Longleftrightarrow$ $L(M)$ is infinite. 

Note also that if we allow $K = \lambda$, we arrive at the previous result where
any automaton $M$ with $L(M) \ne \emptyset$ also accepts the infinite language
${L(M)}^\&$, as we can `pump' any connected DAG $\dd$ to create infinitely many
disconnected copies. 

\section{Algorithms for Intersection and Union}

In this section we will provide algorithms (and hence constructive proofs) that
the set of recognisable weighted DAG languages is closed under intersection and
union. The following algorithms no longer require the automata to be reduced or
deterministic. 

\subsection{Union}
Given weighted DAG automata $M_1$ and $M_2$, we define the union of their
languages $L(M_1)$ and $L(M_2)$, $L = L(M_1) \cup L(M_2)$ such that $L(D) =
[[M_1]](D) \oplus [[M_2]](D)$. 

The following algorithm computes the union automaton, assuming that $Q_1 \cap
Q_2 = \emptyset$.

\begin{algorithm}[!h]
  \caption{$\textsc{Union}(M_1 = \langle \Sigma, Q_1, \Theta_1, \delta_1, \K
  \rangle, M_2 = \langle \Sigma, Q_2, \Theta_2, \delta_2, \K \rangle)$}
	\label{alg:union}
  $Q_M \gets Q_1 \cup Q_2$\\
  $\Theta_M \gets \Theta_1 \cup \Theta_2$\\
  $\delta_M \gets \delta_1 \cup \delta_2$\\
  \Return{$\langle \Sigma, Q_M, \Theta_M, \delta_M, \K\rangle$}
\end{algorithm}

\begin{theorem}
  Given $D \in \calD_\Sigma$, weighted DAG automata $M_1$ and $M_2$, $M =
    \textsc{Union}(M_1, M_2)$ gives us an automaton such that $[[M]](D) =
    [[M_1]](D) \oplus [[M_2]](D)$.
\end{theorem}

\begin{proof}
  Since $Q_1 \cap Q_2 = \emptyset$, we also have $\Theta_1 \cap \Theta_2 =
  \emptyset$. This means that for any transition $\alpha \xrightarrow{a} \beta
  \in \Theta_M$, we have $\alpha \in Q_1^* \Longleftrightarrow \beta \in Q_1^*$.
  This also means that any run $\rho$ with $\delta_M(\rho) \ne \veczero$ must be
  of the form $\rho: E \rightarrow Q_1$ or $\rho: E \rightarrow Q_2$ and thus we
  get:
  
  \begin{align*}
    [[M]](D) &= \bigoplus_{\text{all } \rho \text{ on } D} \delta_M(\rho)\\
    &= \bigoplus_{\rho: E \rightarrow Q_1} \delta_M(\rho) \oplus \bigoplus_{\rho: E \rightarrow Q_2} \delta_M(\rho)\\
    &= \bigoplus_{\text{all } \rho_1 \text{ on } D} \delta_1(\rho_1) \oplus \bigoplus_{\text{all } \rho_2 \text{ on } D} \delta_2(\rho_2)\\
    &= [[M_1]](D) \oplus [[M_2]](D)
  \end{align*}
\end{proof}
As the collection of $\rho: E \rightarrow Q_1$ is exactly the set of runs
$\rho_1$ of $M_1$, and by definition $\delta_M(\alpha \xrightarrow{a} \beta) =
\delta_1(\alpha \xrightarrow{a} \beta)$ when $\alpha,\beta \in Q_1^*$, hence
$\delta_M(\rho_1) = \delta_1(\rho_1)$. The case for
 $\rho: E \rightarrow Q_2$ is symmetric. 

\subsection{Intersection}
Given weighted DAG automata $M_1$ and $M_2$, we define the intersection of their
languages $L(M_1), L(M_2)$, $L = L(M_1) \cap L(M_2)$ such that $L(D) = [[M_1]](D) \otimes
[[M_2]](D)$. 

The following algorithm computes the intersection automaton, assuming that $Q_1 \cap
Q_2 = \emptyset$.

\begin{algorithm}[!h]
  \caption{$\textsc{Intersect}(M_1 = \langle \Sigma, Q_1, \Theta_1, \delta_1, \K
  \rangle, M_2 = \langle \Sigma, Q_2, \Theta_2, \delta_2, \K \rangle)$}
	\label{alg:naiveintersect}
  $Q_M \gets Q_1 \times Q_2$\\
  $\Theta_M \gets \emptyset$\\
  $\delta_M \gets \veczero$\\
  \ForEach{$t_1 = q_1q_2\ldots q_{m_1}\xrightarrow{\sigma_1} r_1r_2 \ldots
  r_{n_1} \in \Theta_1$}{ \ForEach{$t_2 = 
  q_1'q_2'\ldots q_{m_2}' \xrightarrow{\sigma_2} r_1'r_2'\ldots r_{n_2}' 
  \in \Theta_2$}{ \If{$m_1 = m_2, n_1 = n_2, \sigma_1 = \sigma_2$}{ 
    $t_M =
  (q_1q_1')(q_2q_2') \ldots (q_{m_1}q_{m_2}) \xrightarrow{\sigma_1}
  (r_1r_1')(r_2r_2') \ldots (r_{n_1}r_{n_2}')$\\
        $\Theta_M \gets \Theta_M \cup \{t_M\}$\\
        $\delta_M(t_M) = \delta_1(t_1) \otimes \delta_2(t_2)$\\
      } } } \Return{$\langle \Sigma, Q_M, \Theta_M, \delta_M, \K\rangle$}
\end{algorithm}

\begin{theorem}
    Given $D \in \calD_\Sigma$, weighted DAG automata $M_1$ and $M_2$, $M =
    \textsc{Intersect}(M_1, M_2)$ gives us an automaton such that $[[M]](D) =
    [[M_1]](D) \otimes [[M_2]](D)$.
\end{theorem}
\begin{proof}
Let $\rho_M: E \rightarrow Q_M$ be a run of $M$ on $D$. Then, $\rho_{M_1}: E
\rightarrow Q_1, \rho_{M_1}(e) = \rho_M(e)[0]$ and $\rho_{M_2}: E \rightarrow
Q_2, \rho_{M_2}(e) = \rho_M(e)[1]$ are runs of $M_1$ and $M_2$ on $D$
respectively (where $\rho(e)[i]$ refers to the $i$'th element of the tuple
indexed at 0). 

Defining $t_{1v} =  q_1\ldots q_{m_v} \xrightarrow{lab_V(v)}
r_1\ldots r_{n_v} = \langle \rho_{M_1}(in(v)), lab_V(v),
\rho_{M_1}(out(v)) \rangle$ and $t_{2v} =  q_1'\ldots q_{m_v},
\xrightarrow{lab_V(v)} r_1'\ldots r_{n_v}'  = \langle \rho_{M_2}(in(v)),
lab_V(v), \rho_{M_2}(out(v)) \rangle$, the run weight of $\rho_M$ is:

\begin{align*}
  \delta_M(\rho_M) &= \bigotimes_{v \in V} \delta(\langle \rho_M(in(v)), lab_V(v), \rho_M(out(v))\rangle)\\
  &= \bigotimes_{v \in V} \delta(\langle (q_1q_1')(q_2q_2') \ldots (q_{m_v}q_{m_v}'), lab_V(v), (r_1r_1')(r_2r_2') \ldots (r_{n_v}r_{n_v}')\rangle)\\
  &= \bigotimes_{v \in V} \delta_1(t_{1v}) \otimes \delta_2(t_{2v}) &&\text{Line 9}\\
  &= \bigotimes_{v \in V} \delta_1(t_{1v}) \otimes \bigotimes_{v \in V} \delta_2(t_{2v}) = \delta_1(\rho_{M_1}) \otimes \delta_2(\rho_{M_2})\\
\end{align*}

Note that the second last line is only defined if both $t_{1v}$ and $t_{2v}$ are
in $\Theta_1$ and $\Theta_2$, but if this is not the case then by definition the
transition weight is 0. More formally, if one of $t_{iv} \notin \Theta_i$, then
$\delta_i(\rho_{M_i}) =0$ and $\delta_M(\rho_M) = 0$.

Hence, every run $\rho_M$ of $M$ on $D$ has associated with it two runs,
$\rho_{M_1}$ of $M_1$ and $\rho_{M_2}$ of $M_2$, such that we can write
$\delta_M(\rho_M) = \delta_1(\rho_{M_1}) \otimes \delta_2(\rho_{M_2})$ 

Thus, we have:
\begin{align*}
  [[M]](D) &= \bigoplus_{\text{all $\rho$ on $D$}} \delta_M(\rho)\\
  &= \bigoplus_{\rho_{M_1}, \rho_{M_2} \text{ on $D$}} \delta_1(\rho_{M_1}) \otimes \delta_2(\rho_{M_2})\\
  &= \bigoplus_{\rho_{M_1} \text{ on $D$}} \delta_1(\rho_{M_1}) \otimes \bigoplus_{\rho_{M_2} \text{ on $D$}} \delta_2(\rho_{M_2}) &&\text{by distributivity}\\
  &= [[M_1]](D) \otimes [[M_2]](D)
\end{align*}
\end{proof}

\section{Conclusions and Future Work}
In this paper we have examined properties of DAG automata, providing definitions
and properties in line with those of string and tree automata. Specifically, we
have given a more formal definition of a partial weight and graph contexts, and thus proved
correctness of the recognition algorithm presented in \cite{chiang2018weighted}.
We then provided an alternative definition of unweighted state equivalence in
terms of the hypergraph representation of an automaton, which may be a useful
perspective from which to derive a definition of weighted equivalence (e.g by
comparing the weights of the generated graph fragments for two equivalent
states). We also provide a definition of cyclic automata, again based on the
hypergraph representation and a conjecture that an automaton being cyclic means
its language of connected DAGs is infinite. Finally, we provided algorithmic proofs of
closure under weighted union and intersection, a simple generalisation of the
techniques used for weighted string and tree automata.  

We believe that there is significant potential for future work in this area. For
instance, concepts of weighted equivalence, Myhill-Nerode theorems, and weight
pushing can be investigated. Hopefully, this would lead to algorithms for weighted determinisation and
weighted minimisation. 

Training algorithms for weighted DAG automata also represent a valuable area of
research, given the existence of training/weight learning algorithms for string
and tree automata.

Graph grammars are also a well studied topic, gaining traction after the seminal
paper \cite{drewes1997hyperedge} introducting hyperedge replacement graph
grammars. The connection between graph grammars (perhaps the node replacement
variant as described in \cite{rozenberg1997handbook}) and DAG automata may also
be worth exploring, especially the relationship with our definition of graph contexts.


\bibliography{custom}
\bibliographystyle{acl_natbib}

\appendix
\section{Appendix: External Algorithms and Definitions}
\subsection{Recognition/Graph Weight Computation Algorithm} 

We present this algorithm as specified in \cite{chiang2018weighted}. $star(v)$
represents the set of incoming and outgoing edges from (super)vertex $v$. $f|_I$
is the function $f$ restricted to edges in set $I$. 

\begin{algorithm*}[H]
  \caption{$[[M]](D)$ Computation}
  \label{alg:chiangrecognition}
  \ForEach{node $v$}{
    \ForAll{$f:star(v) \rightarrow Q$}{
      $\aalpha[v,f] \gets \delta(\langle f[in(v)], lab(v), f[out(v)]\rangle)$\\
    }
  }
  \ForEach{edge $e$ in order, s.t. $e$ has not been deleted}{
    $(u,v)\gets (src(e), tar(e))$\\
    $I \gets star(u) \cap star(v)$\\
    Create new supervertex $z$\\
    $in(z) \gets in(u) \cup in(v) \setminus I$\\
    $out(z) \gets out(u) \cup out(v) \setminus I$\\
    \ForAll{$h:star(z) \rightarrow Q$}{
      $\aalpha[z,h] \gets 0$\\
    }
    \ForAll{$f:star(u) \rightarrow Q$}{
      \ForAll{$g:star(v) \rightarrow Q$ s.t. $f|_I = g|_I$}{
        $h = f \cup g \setminus f|_I$\\
        $\aalpha[z,h] \gets \aalpha[z,h] \oplus\aalpha[u,f] \otimes \aalpha[v,g]$\\
      }
    }
    Delete $u,v$ and all edges in $I$\\
  }
  One supervertex $v$ remains\\
  \Return{$\aalpha[v,\emptyset]$}
\end{algorithm*}


\subsection{State Equivalence}
We present the definition of equivalent/distinguishable states as specified in
\cite{blum2019properties}.

\begin{definition}[Distinguishable State Pair]
  Let $A=(Q,\Sigma, \Theta)$ be a reduced top down deterministic DAG automaton.
  The set of all distinguishable state pairs is defined inductively as the
  smallest symmetric binary relation $\Delta_A \subseteq Q^2$ such that $(p,p')
  \in \Delta_A$ if there exists $\sigma \in \Sigma$ and $\alpha, \alpha' \in
  Q^*$ such that one of the following hold:
  \begin{enumerate}
    \item $\Theta$ contains a rule $\alpha p \alpha' \xrightarrow{\sigma} ...$
    but no rule $\alpha p' \alpha' \xrightarrow{\sigma} ...$
    \item There are rules $\alpha p \alpha' \xrightarrow{\sigma} q_1\cdots q_n$
    and $\alpha p' \alpha' \xrightarrow{\sigma} q_1'\cdots q_n'$ in $\Theta$
    such that $(q_j,q_j') \in \Delta_A$ for some $j \in [n]$.
  \end{enumerate}
\end{definition}

\begin{definition}[Equivalent States]
  States $p,p' \in Q$ are equivalent, $p\equiv p'$ if $(p,p') \notin \Delta_A$.
\end{definition}

\section{Appendix: Unfinished Work (DAG Grammars)}
\subsection{Weighted Order Preserving DAG Grammars}
\subsubsection{Graph Definition}
\begin{definition}[Hyperedge Labelled, Directed Hypergraph]
  \label{def:grammargraph}
  Let $(\Sigma, rank)$ be a ranked alphabet. A directed, hyperedge labeled
  hypergraph over $\Sigma$ is a graph $G = (V, E, src, tar)$ alongside a
  labelling function $lab_E$. $lab_E: E \rightarrow \Sigma$ is a labelling of
  the edges such that $rank(lab(e)) = |tar(e)|$ for $e \in E$, i.e the rank of
  an edge's label corresponds to the number of targets.  We will express such
  graphs as tuples $(V, E, src, tar, lab_E)$.
\end{definition}
  
Let $\calA_\Sigma$ be the set of all hyperedge labelled DAGs over ranked
alphabet $\Sigma$.

\begin{definition}[Marked Hypergraph]
  \label{def:markedhypergraph}
  A marked hyperedge labelled, directed hypergraph is a tuple $G =
  (V,E,src,tar,lab_E,X)$ where $(V,E,src,tar,lab_E)$ forms a graph as in
  Definition \ref{def:grammargraph} and $X \in V^\circledast$ is a nonempty
  sequence called the \textit{marking} of $G$, and whose nodes are referred to
  as \textit{external nodes}. As $X$ is nonempty, we can write it as $X = (v,w)$
  where $v \in V$ and $w \in V^\circledast$ and say that $v = root(G)$, $w =
  ext(G)$. 
\end{definition}
    

\subsubsection{Preliminaries}
Let $D = (V,E,src,tar,lab_E)$ be a DAG as per Definition \ref{def:grammargraph},
and $u,v \in V$. An edge $e$ is a \textit{common ancestor edge} of $u$ and $v$
if there exists $t, t' \in tar(e)$ such that $u$ is a descendant of $t$ and $v$
is a descendant of $t'$. An edge $e$ is the \textit{closest common ancestor
edge} of $u$ and $v$ if there is no other common ancestor edge $e'$ such that
$src(e') \in tar(e)$. Note: a pair of nodes may have more than one closest
common ancestor edge in a DAG. 

\begin{definition}[Partial Order on Leaves of DAG]
    Let $D = (V,E,src,tar,lab_E)$ be a DAG. We say that $u \preceq_D v$ for $u,v
    \in V$ if for every closest common ancestor edge $e$ of $u$ and $v$, the
    sequence $tar(e)$ can be written as $wtw'$ where $w,w' \in V^\circledast$
    and $t\in V$ such that $t$ is an ancestor of $u$ and any ancestors of $v$
    are in $tw'$. 
\end{definition}
\begin{claim}
For DAG $D$, the transitive closure of $\preceq_D$ forms a partial order on the
leaves of $D$.
\end{claim}
\begin{proof} The relation is reflexive as the closest common ancestor edges of
a node $u$ and itself are the set of incoming edges to $u$, and the targets of
such an edge $e$ can be written $wuw'$. To show antisymmetry, assume that $u
\preceq_D u'$ and $u' \preceq_D u$ for $u,u' \in V$. Then, it must be the case
that for any closest common ancestor edge $e$, $tar(e) = w_1tw_1' = w_2t'w_2'$,
where $t,t'$ are ancestors of $u,u'$ respectively, and $t' \in tw_1'$ and $t \in
t'w_2'$. However, this is only possible if $t = t'$, and since $e$ by assumption
is a closest common ancestor, $t = t' \Longleftrightarrow u = u'$. 

Note that we need to take the transitive closure as $\preceq_D$ is not
necessarily transitive on its own - consider $u \preceq_D u'$, $u' \preceq_D
u''$ for $u,u',u'' \in V$ but $u$ and $u''$ do not share a common ancestor edge. 
\end{proof}

\subsubsection{Definition}
A (weighted) order preserving DAG grammar is a structure $H = (\Sigma, N, P, S,
w, \K)$ where:
\begin{itemize}
    \item $\Sigma$ is a ranked alphabet of terminal symbols
    \item $N$ is a ranked alphabet of nonterminal symbols
    \item $P$ is a set of production rules
    \item $S \in N$ is the starting nonterminal. We define $S^\bullet$ as the
    starting graph with a single edge labelled by $S$ connecting a marked source
    node $v_0$ to $k=rank(S)$ marked targets $v_1,v_2,\ldots v_k$. 
    \item $\K$ is a commutative semiring
    \item $w: P \rightarrow \K$ is a weight function
\end{itemize}

The rules in $P$ must be of the form $A \rightarrow F$ where $A \in N$ and $F$
is a marked DAG with $|ext(F)| = rank(A)$ that obeys one of the following forms:
\begin{enumerate}
    \item $F$ contains just two edges $e_1, e_2$ both labelled with $A$,
    $src(e_1) = src(e_2)$, $tar(e_1) = tar(e_2)$. Further, all nodes are marked
    ($V = \{x \in X\}$) and connected to the two edges. This is called a
    \textit{clone} rule.
    \item $F$ has height at most 2, and obeys the following:
    \begin{itemize}
        \item All nodes have out degree at most one.
        \item There is a single root node, which is the marked $v = root(F)$,
        and the single edge $e$ with $src(e) = v$ is labelled by a terminal
        symbol. 
        \item Every other edge is labelled with a nonterminal symbol.
        \item Only leaves have in-degrees larger than one.
        \item All targets of nonterminal edges have either in-degree larger than
        one or are marked nodes.
        \item $\preceq_F$ gives a total ordering of the leaves and $ext(F)$
        respects this ordering (i.e the sequence $ext(F)$ is in `increasing'
        order with respect to $\preceq_F$). 
    \end{itemize}
\end{enumerate}

These restrictions are harsh but are ensure polynomial parsing.
\citeauthor{ericson1opdl} show how relaxing some of these restrictions leads to
an NP-Complete membership problem, but a full characterisation of strictly
necessary restrictions for polynomial parsing remains an open issue. 

A derivation step of $H$ on graph $G$, given rule $\rho = A \rightarrow F$,
takes an edge $e \in E_G$ labelled with $A$ and replaces it with the marked DAG
$F$. $src(e)$ is identified with $root(F)$, and the length $k = rank(A)$
sequence $tar(e)$ is identified with the sequence $ext(F)$ (i'th node of
$tar(e)$ is i'th node of $ext(F)$).  If $G'$ is the resultant graph, we say $G
\Rightarrow_\rho G'$. We write $\Rightarrow_H^*$ to denote the reflexive
transitive closure of $\Rightarrow$, such that $G \Rightarrow_H^* G'$ if $G'$
can be derived in zero or more derivation steps using rules of $H$. 

We denote by $L(H)$ the \textbf{language} of grammar $H$, where $L(H) = \{g \in
\calA_\Sigma \, | \, S^\bullet \Rightarrow_H^* g \}$, i.e the set of terminal
labelled graphs generated from the starting graph.

In the weighted case, we also overload the weight function to define the
\textbf{derivation weight} of a derivation $\tt = S^\bullet \Rightarrow_{p_0}
G_0 \Rightarrow_{p_1} G_1 \ldots \Rightarrow_{p_k} G_k$ as:
\[
    w(\tt) = \bigotimes_i w(p_i)
\]

The \textbf{graph weight} of $D \in \calA_\Sigma$ under WOPDG $H$, denoted by
$\calS_H$ is:

\[
    \calS_H(D) = \bigoplus_{\text{all distinct $\tt$}} w(\tt)
\]
Where two derivations are distinct if their independent derivation steps cannot
be reordered to be the same. Two derivation steps $p_1,p_2$ are independent if
for $G, G'$ such that $G \Rightarrow_{p_1} G_1 \Rightarrow_{p_2} G'$, we also
have $G \Rightarrow_{p_2} G_2 \Rightarrow_{p_1} G'$. 


\end{document}
